\documentclass{Academic}

\begin{document}
%Easy customisation of title page
%TC:ignore
\myabstract{\small
\noindent\textbf{Abstract}
%Abstract below

Dalam tulisan kali ini, saya akan menuliskan bagaimana cara secara singkat untuk bekerja menggunakan Overleaf dan GitHUb. Harapan saya adalah supaya aku dapat ingat dan menggunakan nya untuk kepentingan penelitian.}
\renewcommand{\myTitle}{Catatan dan tatacara menggunakan git dan github}
\renewcommand{\MyAuthor}{Enggar Alfianto}
\renewcommand{\MyDepartment}{Graduate School of Engineering Osaka University}
\renewcommand{\ID}{1234567}
\renewcommand{\Keywords}{Keyword 1, Keyword 2, Keyword 3}
\maketitle
%\vspace{-1.9em}\noindent\rule{\textwidth}{1pt} %add this line if not using abstract
\onehalfspacing
%TC:endignore

\section{Memulai}

\begin{enumerate}
    \item Buatlah akun pada github
    \item Buat repository kosong.
    \item Akan muncul perintah awal di laman tersebut
    \item Buat file di laptop (fileA.tex).
    \item Gunakan perintah berikut di laptop:
    \begin{enumerate}
        \item \textbf{git init}
        \item \textbf{git add fileA.tex}
        \item \textbf{git commit -m "first-commit"}
        \item \textbf{git branch -M main}
        \item \textbf{git remote add origin https://github.com/enggaralfi/1.git}
        \item \textbf{git push -u origin main} \label{push}
        \item \textbf{git add file} file yang mau di add kemudian diakhiri dengan \textbf{git commit -m "isian komit nya"}
        \item kemudian kembali ke langkah \ref{push}.
        \item Jika dalam github sudah ada perubahan, maka perlu di sinkronkan dulu dengan cara \textbf{git pull origin}.
    \end{enumerate}
\end{enumerate}

\section{Git push dan pull}
Apa perbedaan git push dan git pull?

\subsection{git push}
Seperti namanya git push adalah untuk mendorong. Artinya mendorong apa yang sudah diubah ke repository online.

\subsection{git pull}
Sedangkan git pull adalah untuk menarik. Yaitu menarik perubahan yang ada di online repository ke dalam laptop kita.

\section{Conclusion}
Berdasarkan hasil pengamatan dan pembelajaran, menggunakan overleaf dan latex sangatnyaman dan memudahkan pekerjaan kita \cite{einstein}.

% There is a brief and precise description of the context of the work such that it is easy to understand the significance of the conclusions.  
% Each conclusion is described precisely, and correlates exactly with the evidence discussed in the discussion section.
% The significance of the work is precisely described.
% There is no new information that has not been discussed in the rest of the report.
% The conclusions section correlates precisely with the abstract, and is easy to understand if taken out of context.


%TC:ignore
%\clearpage %add new page for references
\singlespacing
\emergencystretch 3em
\hfuzz 1px
\printbibliography[heading=bibnumbered]
% \bibliography{references}

% \clearpage
% \begin{appendices}

% \section{Here go any appendices!}

% \end{appendices}

%TC:endignore
\end{document}